\paragraph{18}
\subparagraph{a}
Mit der bedingten Wahrscheinlichkeit folgt:
\begin{equation}
P(F|W) = \frac{P(F\cap\ W)}{P(W)} = \frac{\frac{P(F\,\cap\, W)}{P(F)}\cdot P(F)}{P(W)} = \frac{P(W|F)\cdot P(F)}{P(W)}
\end{equation}

\subparagraph{b)}
Mit
\begin{equation}
  P(F)=9/14
\end{equation}
folgt mit der Wahrscheinlichkeit der gesuchten Attributen in $F$
\begin{align}
%Wahrscheinlichkeit für die geg. Parameter in F:
P_{Wind_F}=\frac{1}{3} \\
P_{Temp_F}=\frac{1}{3} \\
P_{Feucht_F}=\frac{1}{3} \\
P_{Aussicht_F}=\frac{2}{9}
\end{align}
und mit der Gesamtwahrscheinlichkeit der Ereignisse
\begin{align}
  P_{Wind_{Stark}}=\frac{6}{14} \\
  P_{Temp_{Kalt}}=\frac{6}{14} \\
  P_{Aussicht_{Sonnig}}=\frac{3}{14} \\
  P_{Feucht_{Hoch}}=\frac{7}{14} \\
\end{align}
und dem in der Aufgabenstellung gegebenen Zusammenhang
\begin{equation}
  P(F| W)=\prod_{i} P(x_i | W)
\end{equation} der Zusammenhang
\begin{align}
  P(F | W) =& \frac{P_{Wind_F} \cdot P_{Temp_F} \cdot P_{Feucht_F} \cdot P_{Aussicht_F}\cdot P(F)}{P_{Wind_{Stark}} \cdot P_{Temp_{Kalt}} \cdot P_{Aussicht_{Sonnig}} \cdot P_{Feucht_{Hoch}}} \\
  P(F | W) =& 26,9 \%
\end{align}
.

\subparagraph{c)}
Die Wahrscheinlichkeit beträgt 0, da der Datensatz keine Aufzeichnungen von stattgefundenden Spielen bei heißen Temperaturen enthält.
Dies erzeugt einen Faktor 0, der das gesamte Produkt 0 werden lässt.
Ein größerer Datensatz könnte das Problem beheben.
